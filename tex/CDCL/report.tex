% !TeX encoding = UTF-8
% !TeX program = xelatex+shellescape
% !TeX spellcheck = LaTeX
%
% Author : Shlw
\documentclass[a4paper,10pt]{article}

\usepackage{amsmath}
\usepackage{graphicx}
\usepackage{hyperref}
\usepackage{geometry}
\usepackage{caption}
\usepackage{minted}
\usepackage{xcolor}
\usepackage{dingbat}

\geometry{left=1.5cm,right=1.5cm,top=1.5cm,bottom=2cm}

\hypersetup{hidelinks}

\definecolor{bg}{rgb}{0.95,0.95,0.95}
\renewcommand{\listingscaption}[1]{\begin{center}CODE: #1\end{center}\vspace{-20pt}}
\setminted{
bgcolor=bg, frame=lines,
mathescape, breaklines, breakanywhere,
breaksymbolleft=\raisebox{0.8ex}{\small\reflectbox{\carriagereturn}},
breaksymbolright=\small\carriagereturn
}

\title{DPLL Based SAT-Solver}
\author{Kevin Wu\\1600012832}
\date{\today}

\begin{document}
\maketitle
\vspace{-20pt}
\section{Key Implementation Ideas}
\begin{enumerate}
\setlength{\itemsep}{.1em}
%\item \textbf{Parser:} By \mintinline{python}|input().split()|,
    %I could easily split those words. To provide convenient debug 
    %way, the \textit{CNFparser.py} also accepts inputs from terminal.
    %To maintain robustness and alert when input is incorrect,
    %format checkers are also added in it.
    %You can see different error types at the head of \textit{CNFparser.py}.
\item \textbf{Data Structure:} The way I chose to express clauses is to have $D$  
    \textit{a dictionary of variable to the set of equation index it appears}, $E$ a 
    \textit{list of equations which are sets of variables}, and $L$ \textit{a list of current
    equation index}.
\item \textbf{Preparation:}
    In \textit{DPLL}, my program will not check whether there exists both $X$ and
    $\neg X$ in the same clause. So I checked this in my preparation part, which can
    save some search time.
\item \textbf{Basic Property I:} Any time in the program running, it is guaranteed
    that the $3$ data structures above
    only represent current situation, which means eliminated variables and 
    equations will not appear in any of them.
\item \textbf{Set Variable:} \textit{Set Variable} is a fundamental function for \textit{BCP},
    \textit{Set Pure}, and \textit{Decision}.
    To keep \textit{Property I}, when using $D$ to go through
    the equation $E_k$ where variable $v$ appears,
    I need to check two thing: 
    \begin{enumerate}
        \small\setlength{\itemsep}{.1em}
        \item If the equation becomes empty, I need to eliminate it in $L$.
        \item If the equation becomes true, I need to eliminate it in $L$,
            and remove all the other variables $v'$ in this equation which clears $E_k$,
            and get $D[v']$ rid of $k$.
    \end{enumerate}
\item \textbf{Basic Property II:} Whenever program goes into \textit{Set Variable}, it is 
    assumed that the assignment will not generate conflicts. Thus \textit{Set Variable} does
    not check satisfiability. Moves described below will guarantee this.
\item \textbf{BCP:} \textit{BCP} needs to go through $E$ to single out those one literal
    equation. Use a \textit{set} to store them. When one is added, check whether 
    its negation is inside. There is a potential trap in it depending how you implement 
    \textit{BCP}, I will describe it below.
\item \textbf{Set Pure:} \textit{Set Pure} needs to go through $D$ to pick those literals
    the negation of which does not appear in equations. After this, set them to $true$,
    which will maximumly preserve the satisfiability.
\item \textbf{Decision:} \textit{Decision} is what program will do after
    \textit{BCP} and \textit{Set Pure}. The way to pick what variable to try the assignment
    is pretty tricky. I tried several strategies,
    such as \textit{Random Pick}, \textit{Frequent
    Pick}, \textit{Balanced Pick\footnote{\textit{Balanced Pick} means
    the difference between it and its negation is as small as possible. It indicates either
    $true$ or $false$ assignment is balanced in \textit{DFS}.}}. It turns out
    \textit{Pick the most frequently appeared} is about $30\%$ faster than \textit{Random
    Pick} and way better than \textit{Balanced Pick}. 
    \textit{Frequent Pick} and \textit{Balanced Pick} combined together
    can not provide substantial improvement. 
\item \textbf{DPLL:} \textit{DPLL} is basically the combination of the algorithms above. 
    To maintain \textit{Property I}, I need to fix change when program recalls.
    Considering
    efficiency, I have $2$ set of \textit{lists}; one records changes from \textit{BCP} and
    \textit{Set Pure}, one records changes from \textit{Decision}. When program backtracks
    from the first assignment trial,
    undo the \textit{Decision} change; when it returns from 
    the second trial, undo all the change and return to parent function.
\end{enumerate}
\section{Potential Traps}
\begin{enumerate}
\setlength{\itemsep}{.1em}
\item In \textit{BCP}, there is a case needs further consider. Assume after you go through
    all the clauses containing one literal, these literals are set $true$ at
    the same time without further consideration, there will be a potential bug.
    Considering this case $\{X\},\{\neg X,\neg Y\},\{Y\}$. If you set $X=true$ and $Y=true$
    without checking the clause $\{\neg X,\neg Y\}$, it will be satisfiable, which is
    incorrect. So in my version, I set them one by one.
\item Pay attention to Python Import. Name file and function carefully.
    I was bothered by the strange error log
    when I have file like \textit{string.py} or \textit{parser.py}
    with function \textit{parse} in it.
\item I used test data from UBC 
    website\footnote{\url{http://www.cs.ubc.ca/~hoos/SATLIB/benchm.html}}.
    In these data, there is a mysterious \% at the end of every file input.
    So, I had to add a special check for this in my \textit{CNFparser.py}.
\end{enumerate}
\textbf{Note:} Aside from DPLL based SAT-Solver, I also implemented a CDCL based SAT-Solver, the code and report of which can be found in \url{this website}
\end{document}
